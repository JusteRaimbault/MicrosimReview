%% BioMed_Central_Tex_Template_v1.06
%%                                      %
%  bmc_article.tex            ver: 1.06 %
%                                       %

%%IMPORTANT: do not delete the first line of this template
%%It must be present to enable the BMC Submission system to
%%recognise this template!!

%%%%%%%%%%%%%%%%%%%%%%%%%%%%%%%%%%%%%%%%%
%%                                     %%
%%  LaTeX template for BioMed Central  %%
%%     journal article submissions     %%
%%                                     %%
%%          <8 June 2012>              %%
%%                                     %%
%%                                     %%
%%%%%%%%%%%%%%%%%%%%%%%%%%%%%%%%%%%%%%%%%


%%%%%%%%%%%%%%%%%%%%%%%%%%%%%%%%%%%%%%%%%%%%%%%%%%%%%%%%%%%%%%%%%%%%%
%%                                                                 %%
%% For instructions on how to fill out this Tex template           %%
%% document please refer to Readme.html and the instructions for   %%
%% authors page on the biomed central website                      %%
%% http://www.biomedcentral.com/info/authors/                      %%
%%                                                                 %%
%% Please do not use \input{...} to include other tex files.       %%
%% Submit your LaTeX manuscript as one .tex document.              %%
%%                                                                 %%
%% All additional figures and files should be attached             %%
%% separately and not embedded in the \TeX\ document itself.       %%
%%                                                                 %%
%% BioMed Central currently use the MikTex distribution of         %%
%% TeX for Windows) of TeX and LaTeX.  This is available from      %%
%% http://www.miktex.org                                           %%
%%                                                                 %%
%%%%%%%%%%%%%%%%%%%%%%%%%%%%%%%%%%%%%%%%%%%%%%%%%%%%%%%%%%%%%%%%%%%%%

%%% additional documentclass options:
%  [doublespacing]
%  [linenumbers]   - put the line numbers on margins

%%% loading packages, author definitions

%\documentclass[twocolumn]{bmcart}% uncomment this for twocolumn layout and comment line below
\documentclass{bmcart}

%%% Load packages
%\usepackage{amsthm,amsmath}
%\RequirePackage{natbib}
%\RequirePackage[authoryear]{natbib}% uncomment this for author-year bibliography
%\RequirePackage{hyperref}
\usepackage[utf8]{inputenc} %unicode support
%\usepackage[applemac]{inputenc} %applemac support if unicode package fails
%\usepackage[latin1]{inputenc} %UNIX support if unicode package fails


%%%%%%%%%%%%%%%%%%%%%%%%%%%%%%%%%%%%%%%%%%%%%%%%%
%%                                             %%
%%  If you wish to display your graphics for   %%
%%  your own use using includegraphic or       %%
%%  includegraphics, then comment out the      %%
%%  following two lines of code.               %%
%%  NB: These line *must* be included when     %%
%%  submitting to BMC.                         %%
%%  All figure files must be submitted as      %%
%%  separate graphics through the BMC          %%
%%  submission process, not included in the    %%
%%  submitted article.                         %%
%%                                             %%
%%%%%%%%%%%%%%%%%%%%%%%%%%%%%%%%%%%%%%%%%%%%%%%%%


%\def\includegraphic{}
%\def\includegraphics{}



%%% Put your definitions there:
\startlocaldefs
\endlocaldefs


%%% Begin ...
\begin{document}



\title{A citation network analysis of a scientific landscape in urban modeling\\\medskip
\textit{Abstract proposal}\medskip\\
\textit{Applied Network Science}\\
\textit{Special Issue Complexity and the City}
}
\author{\noun{Juste Raimbault}$^{1,2,3}$\medskip\\
$^1$ UPS CNRS 3611 ISC-PIF\\
$^2$ CASA, UCL\\
$^3$ UMR CNRS 8504 Géographie-cités
}
%\date{Novembre 2016}
\date{}

\maketitle

\justify


%\begin{abstract}
%\end{abstract}

\pagenumbering{gobble}

%\vspace{-1cm}

\textbf{Keywords: }\textit{Citation network; Interdisciplinarity; Urban Modeling; Land-use Transport Interaction Models; Microsimulation Models}

\bigskip

Several modeling approaches to the evolution of urban and territorial environments have been introduced in the literature, providing complementary viewpoints and methods. This paper contributes to the mapping of this scientific landscape by a citation network analysis. Taking the case study of two landmark types of operational models, namely microsimulation models and land-use transport interaction model, we construct an initial corpus through keyword requests. The backward citation network is then reconstructed up to depth two, yielding a network with $1.47\cdot 10^5$ nodes and $2.10\cdot 10^5$ citation links. Community detection in the network provides endogenous disciplines and approaches, including a broader scope than the initial corpus, with for example spatial statistics and agent-based modeling. We construct measures of interdisciplinarity to identify bridge papers, and investigate possible missing connections. We illustrate these more particularly with the interface of microsimulation, agent-based modeling and land-use transport models, and discuss potential coupling of models filling this gap. We show therein how potential research directions can be unveiled through citation network analysis and provide partial maps of science in the field of urban modeling, which can be used by researchers and practitioners to better contextualize their modeling approaches.





%%%%%%%%%%%%%%%%%%%%
%% Biblio
%%%%%%%%%%%%%%%%%%%%

%\footnotesize

%\bibliographystyle{apalike}
%\bibliography{biblio}


\end{document}
